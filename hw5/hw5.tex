\documentclass[12pt,letterpaper]{article}

\newenvironment{proof}{\noindent{\bf Proof:}}{\qed\bigskip}

\newtheorem{theorem}{Theorem}
\newtheorem{corollary}{Corollary}
\newtheorem{lemma}{Lemma} 
\newtheorem{claim}{Claim}
\newtheorem{fact}{Fact}
\newtheorem{definition}{Definition}
\newtheorem{assumption}{Assumption}
\newtheorem{observation}{Observation}
\newtheorem{example}{Example}
\newcommand{\qed}{\rule{7pt}{7pt}}

\newcommand{\assignment}[4]{
\thispagestyle{plain} 
\newpage
\setcounter{page}{1}
\noindent
\begin{center}
\framebox{ \vbox{ \hbox to 6.28in
{\bf CS446: Machine Learning \hfill #1}
\vspace{4mm}
\hbox to 6.28in
{\hspace{2.5in}\large\mbox{Problem Set #2}}
\vspace{4mm}
\hbox to 6.28in
{{\it Handed Out: #3 \hfill Due: #4}}
}}
\end{center}
}

\newcommand{\solution}[4]{
\thispagestyle{plain} 
\newpage
\setcounter{page}{1}
\noindent
\begin{center}
\framebox{ \vbox{ \hbox to 6.28in
{\bf CS446: Machine Learning \hfill #4}
\vspace{4mm}
\hbox to 6.28in
{\hspace{2.5in}\large\mbox{Problem Set #3}}
\vspace{4mm}
\hbox to 6.28in
{#1 \hfill {\it Handed In: #2}}
}}
\end{center}
\markright{#1}
}

\newenvironment{algorithm}
{\begin{center}
\begin{tabular}{|l|}
\hline
\begin{minipage}{1in}
\begin{tabbing}
\quad\=\qquad\=\qquad\=\qquad\=\qquad\=\qquad\=\qquad\=\kill}
{\end{tabbing}
\end{minipage} \\
\hline
\end{tabular}
\end{center}}

\def\Comment#1{\textsf{\textsl{$\langle\!\langle$#1\/$\rangle\!\rangle$}}}


\usepackage{amssymb,amsmath}
\usepackage[]{units}
\oddsidemargin 0in
\evensidemargin 0in
\textwidth 6.5in
\topmargin -0.5in
\textheight 9.0in

\begin{document}

\solution{Nikhil Unni}{\today}{5}{Fall 2014}


\pagestyle{myheadings}  % Leave this command alone

\begin{enumerate}
\item SVM
  \begin{enumerate}
  \item [(a)]
    \begin{enumerate}  
    \item [1.]
      $\textbf{w} = [-1,0]^T$\\
      $\theta = 0$
    \item [2.]
      $\textbf{w} = [-0.5,0.25]^T$\\
      $\theta = 0$
    \item [3.]
      I found the two closest positive/negative points, $[(-1.2,1.6), +], [(2,0), -]$, and found the slope between them, $\frac{1.6}{-3.2} = -\frac{1}{2}$, and the midpoint, $(0.4, 0.8)$, so the line with the farthest distance between the two points (the support vectors), has a slope of $2$ with a point $(0.4, 0.8)$, giving the line $y=2x$, which gives $w=[-2,1]^T, \theta=0$.\\\\
      Then, I just minimized $w$ by halving it repeatedly, until I got $w=[-0.5,0.25]$. This $w$ gave $y(w^Tx+\theta)=1$ for both support vectors, so I know this is the smallest value of $w$ I can get.      
    \end{enumerate}
  \item [(b)]
    \begin{enumerate}
    \item [1.]
      $I = \left\{1,6\right\}$
    \item [2.]
      $\alpha = \left\{\frac{5}{32},\frac{5}{32}\right\}$
    \item [3.]
      Objective function value = $\frac{5}{32}$.
    \end{enumerate}
  \item [(c)]
    $C$ represents how much the SVM should avoid misclassifications. In general, $C$ controls the relative importance of maximizing the margin. For $C=\infty$, we obtain our original hyperplane that we found in (a)-2. For $C=1$, we get a larger margin, with a higher chance of misclassification. The support vectors for $C=1$ can now be inside the margins. For $C=0$ has an even wider margin, with even larger misclassification.
  \end{enumerate}
\item Kernels
  \begin{enumerate}
    \item[(a)]
      \begin{algorithm}
        1. Initialize $\alpha$ to $\vec{0}$ of length $n$, where $n$ is the number of examples.\\
        2. Initialize $\theta$ to $0$.\\
        3. While still making mistakes (terminate after long string of successes):\\
        4. \>\>For each training example $(x,y)$:\\
        5. \>\>\>if $y[(\sum\limits_{i=1}^n\alpha_i y_i \langle x_i, x \rangle) + \theta] < 0$: ($\langle a,b \rangle$ representing the inner product)\\
        6. \>\>\>\>$\alpha_i \leftarrow \alpha_i + 1$ (where i is the index of the current example $(x,y)$)\\
        7. \>\>\>\>$\theta \leftarrow \theta + y$\\
      \end{algorithm}
    \item[(b)]
      $$K(x,z) = \alpha K_1(x,z) + \beta K_2(x,z)$$
      Since $K_1$ and $K_2$ are both valid kernel functions, they can be represented as the dot product of two feature maps, $\phi_1$ and $\phi_2$ such that
      $$K(x,z) = \alpha \langle \phi_1(x), \phi_1(z) \rangle + \beta \langle \phi_2(x), \phi_2(z) \rangle$$
      And also such that
      $$\phi_1(x) = [\phi_1(x)_1, \phi_1(x)_2, \ldots, \phi_1(x)_M]$$
      $$\phi_2(x) = [\phi_2(x)_1, \phi_2(x)_2, \ldots, \phi_2(x)_N]$$
      Where $M$ and $N$ the size of the vectors $\phi_1(x)$ and $\phi_2(x)$ respectively.
      Then, we can represent $K(x,z)$ by using the definition of $\phi_1$ and $\phi_2$ and expand out the inner products
      $$K(x,z) = \alpha \sum\limits_{i=1}^M \phi_1(x)_i \phi_1(z)_i + \beta \sum\limits_{j=1}^N \phi_2(x)_j \phi_2(z)_j$$
      Then we put $\alpha$ and $\beta$ inside the summations as follows
      $$K(x,z) = \sum\limits_{i=1}^M (\sqrt\alpha \phi_1(x)_i) (\sqrt\alpha \phi_1(z)_i) + \sum\limits_{j=1}^N (\sqrt\beta \phi_2(x)_j) (\sqrt\beta \phi_2(z)_j)$$
      Suppose we had a feature map like :
      $$\phi(x) = [\sqrt\alpha \phi_1(x)_1, \sqrt\alpha \phi_1(x)_2, \ldots, \sqrt\alpha \phi_1(x)_M, \sqrt\beta \phi_2(x)_1, \ldots, \sqrt\beta \phi_2(x)_N]$$
      of dimension $N+M$.\\
      Then, $\langle \phi(x), \phi(z) \rangle = \alpha \phi_1(x)_1 \phi_1(z)_1 + \alpha \phi_1(x)_2 \phi_1(z)_2 + \ldots \alpha \phi_1(x)_M \phi_1(z)_M + \beta \phi_2(x)_1 \phi_2(z)_1 + \ldots + \beta \phi_2(x)_N \phi_2(z)_N$\\
      Or 
      $$\langle \phi(x), \phi(z) \rangle = \sum\limits_{i=1}^M (\sqrt\alpha \phi_1(x)_i) (\sqrt\alpha \phi_1(z)_i) + \sum\limits_{j=1}^N (\sqrt\beta \phi_2(x)_j) (\sqrt\beta \phi_2(z)_j)$$
      $$\langle \phi(x), \phi(z) \rangle = K(x,z)$$
      Because $K(x,z)$ is an inner product of our new feature map, it is a valid kernel for all valid kernels $K_1$ and $K_2$ and all positive $\alpha$ and $\beta$.\\
    \item[(c)]
      Before proving that $K(x,z)$ is a valid kernel, I'll prove one more property of kernels.\\
      \textbf{Note}: I'm going to reuse $K$, $x$ and $z$ for this proof, they're not necessarily the same $K$, $x$ and $z$ we were given in the problem, sorry for the slight abuse of notation!\\
      \begin{enumerate}
        \item[1.] $K(x,z) = K_1(x,z)K_2(x,z)$, for all valid kernels $K_1$ and $K_2$\\
          $$K(x,z) = (\langle \phi_1(x), \phi_1(z) \rangle) * (\langle \phi_2(x), \phi_2(z))$$
          $$K(x,z) = (\sum\limits_{i=1}^M \phi_1(x)_i \phi_1(z)_i)(\sum\limits_{j=1}^N \phi_2(x)_j \phi_2(z)_j)$$
          $$K(x,z) = \sum\limits_{i=1}^M \sum\limits_{j=1}^N \phi_1(x)_i \phi_1(z)_i \phi_2(x)_j \phi_2(z)_j$$
          Let $\phi(x)_{ij} = \phi_1(x)_i * \phi_2(x)_j$ and $\phi(z)_{ij} = \phi_1(z)_i * \phi_2(z)_j$. Then:
          $$K(x,z) = \sum\limits_{i=1}^M \sum\limits_{j=1}^N \phi(x)_{ij} \phi(z)_{ij}$$
          $$K(x,z) = \langle \phi(x), \phi(z) \rangle$$
          Where the dimension of $\phi$ is $M*N$.\\
          Because $K$ is the inner product of the feature map of $x$ and $z$, it's a valid kernel.
      \end{enumerate}
      Now that we have that, we can continue with the original proof.\\\\
      First of all, $K_1(x,z) = x^Tz$ is clearly a valid kernel, where the feature map $\phi_1(x)$ is just the identity feature map, so that $K_1 = \langle \phi_1(x), \phi_1(z) \rangle$.\\\\
      Next, because of my proof right above, $(x^Tz)(x^Tz) = (x^Tz)^2$ is a valid kernel too, since it's just the product of two valid kernels. By the same logic, $(x^Tz)(x^Tz)^2 = (x^Tz)^3$ is also a valid kernel.\\\\
      Then, by the proof from part (b), we know that the linear combination (with positive coefficients) of valid kernels is a valid kernel as well.\\
      So $1(x^Tz)^3 + 400(x^Tz)^2$ is valid, and then $1(x^Tz)^3 + 400(x^Tz)^2 + 100x^Tz$ is a valid kernel as well. $\blacksquare$
      
  \end{enumerate}
\item Boosting

        \begin{table}[!t]
      {\centering
        \begin{tabular}{|c|c||c|c|c|c||c|c|c|c|}

          \hline
          & & \multicolumn{4}{c||}{Hypothesis 1}
	  & \multicolumn{4}{c|}{Hypothesis 2} \\
          \cline{3-10}
          {\em i} & Label & $D_0$ & $x_1 \equiv $ & $x_2 \equiv $ & $h_1\equiv$ & $D_1$ &  $x_1 \equiv $ & $x_2 \equiv $ & $h_2 \equiv $ \\
          & & & [$x >5$] & [$y > 6$] & [$x_1$] & & [$x > 3$] & [$y > 8$] & [$x_2$] \\

          \tiny{(1)} & \tiny{(2)} & \tiny{(3)} & \tiny{(4)} &  \tiny{(5)} & \tiny{(6)} & \tiny{(7)} & \tiny{(8)} & \tiny{(9)} & \tiny{(10)}\\
          \hline \hline
          {\em 1} & $-$ &$\nicefrac{1}{10}$ &$-$ &$+$ &$-$ &$\nicefrac{1}{16}$ &$-$ &$+$ &$+$  \\
          \hline                                                                             
          {\em 2} & $-$ &$\nicefrac{1}{10}$ &$-$ &$-$ &$-$ &$\nicefrac{1}{16}$ &$+$ &$-$ &$-$  \\
          \hline                                                                             
          {\em 3} & $+$ &$\nicefrac{1}{10}$ &$+$ &$+$ &$+$ &$\nicefrac{1}{16}$ &$+$ &$-$ &$-$ \\
          \hline                                                                             
          {\em 4} & $-$ &$\nicefrac{1}{10}$ &$-$ &$-$ &$-$ &$\nicefrac{1}{16}$ &$+$ &$-$ &$-$ \\
          \hline                                                                             
          {\em 5} & $-$ &$\nicefrac{1}{10}$ &$-$ &$+$ &$-$ &$\nicefrac{1}{16}$ &$-$ &$+$ &$+$ \\
          \hline                                                                             
          {\em 6} & $+$ &$\nicefrac{1}{10}$ &$+$ &$+$ &$+$ &$\nicefrac{1}{16}$ &$+$ &$-$ &$-$ \\
          \hline                                                                             
          {\em 7} & $+$ &$\nicefrac{1}{10}$ &$+$ &$+$ &$+$ &$\nicefrac{1}{16}$ &$+$ &$+$ &$+$ \\
          \hline                                                                             
          {\em 8} & $-$ &$\nicefrac{1}{10}$ &$-$ &$-$ &$-$ &$\nicefrac{1}{16}$ &$+$ &$-$ &$-$ \\
          \hline                                      
          {\em 9} & $+$ &$\nicefrac{1}{10}$ &$-$ &$+$ &$-$ &$\nicefrac{1}{4}$ &$+$ &$+$ &$+$ \\
          \hline
          {\em 10} & $-$ &$\nicefrac{1}{10}$ &$+$ &$+$ &$+$ &$\nicefrac{1}{4}$ &$+$ &$-$ &$-$ \\
          \hline
        \end{tabular}
        \caption{Table for Boosting results} \label{table:ltu}}
    \end{table}      
  \begin{enumerate}
    \item[(c)]
      For $t=0$ mistakes were made on the $9^{th}$ and $10^{th}$ examples, giving $\epsilon_0 = 0.2$.\\
      Next, $\alpha_0 = \frac{1}{2}log_2(\frac{1-\epsilon_0}{\epsilon_0}) = \frac{1}{2}log_2(4) = 1$\\
      $$D_1 = \frac{D_0(i)}{z_t} * 2^{-\alpha_0} \text{ if } y_i = h_0(x_i)$$
      $$D_1 = \frac{D_0(i)}{z_t} * 2^{\alpha_0} \text{ if } y_i \neq h_0(x_i)$$
      So then, for the first 8 examples, $D_1 = \frac{\frac{1}{10}}{z_t} * 2^{-1}$, and for the last 2, $D_1 = \frac{\frac{1}{10}}{z_t} * 2^{1}$\\
      This means $\frac{8}{20z_t} + \frac{2}{5z_t} = 1$, making $z_t = \frac{4}{5}$.\\
      All together, for the first 8 examples, $D_1 = \frac{1}{16}$, and for the last two, $D_1 = \frac{1}{4}$.
    \item[(d)]
      Mistakes were made on examples 1, 3, 5, and 6, giving $\epsilon_1 = \frac{4}{16} = \frac{1}{4}$.\\
      So then $\alpha_1 = \frac{1}{2}log_2(\frac{1-0.25}{0.25}) \approx 0.79248$.\\
      This makes the final $H(x) = sgn[1(x>5) + 0.79248(y>8)]$.
    \item[(e)]
      The $\epsilon_0$ be the original error over the distribution $D_t$. It's just the sum of all $D_t(i)$ over all indeces where there was a misclassification.
      $$\epsilon_0 = \sum\limits_{i\in I} D_t(i) \text{, where I is the set of all indices where there was a misclassification.}$$
      Then, let $\epsilon_1$ be the new error of the previous hypothesis at time $t$ over the new distribution $D_{t+1}$. Because we have the same hypothesis, we'll be iterating over the same indices as before. So:
      $$\epsilon_1 = \sum\limits_{i\in I} D_{t+1}(i)$$
      Going by the update formula, keeping in mind these examples are all misclassifications, such that the signs are all the same:
      $$\epsilon_1 = \sum\limits_{i\in I} \frac{D_t(i)}{z_t} * e^{\frac{1}{2}ln(\frac{1-\epsilon_0}{\epsilon_0})}$$
      $$\epsilon_1 = \sum\limits_{i\in I} \frac{D_t(i)}{z_t}(\frac{1-\epsilon_0}{\epsilon_0})^{\frac{1}{2}}$$
      $$\epsilon_1 = \frac{\sum\limits_{i\in I} D_t(i)}{z_t}(\frac{1-\epsilon_0}{\epsilon_0})^{\frac{1}{2}}$$
      $$\epsilon_1 = \frac{\epsilon_0}{z_t}(\frac{1-\epsilon_0}{\epsilon_0})^{\frac{1}{2}}$$
      $$\epsilon_1 = \frac{\epsilon_0}{2[\epsilon_0(1-\epsilon_0)]^{\frac{1}{2}}}(\frac{1-\epsilon_0}{\epsilon_0})^{\frac{1}{2}}$$
      $$\epsilon_1 = \frac{\epsilon_0}{2 * \sqrt(\epsilon_0)\sqrt(\epsilon_0)}$$
      $$\epsilon_1 = \frac{1}{2} \blacksquare$$
      
  \end{enumerate}
  \item Probability
    \begin{enumerate}
      \item[(a)]
        \begin{enumerate}
          \item[i.]
            The expected number of children per family in town A is just $1$ since they always stop having children after having their first one.\\
            In town B, the $P(X=1)$ is the probability of having a boy on the first go, which is $0.5$, and $P(X=2)$ is the probability that they had a girl first, then a boy, which is $0.5*0.5$. In general, $P(X=n) = 0.5^n$. So $E[X] = \sum\limits_{i=1}^\infty i * 0.5^i = 2$. So the expected number of children per family in town B is $2$.
          \item[ii.]
            Let $X$ be the random variable denoting number of boys, and $Y$ denoting number of girls.\\
            The expected number of boys per family in town A is just $0.5$, and the expected number of girls is $0.5$ as well. This is because they only have 1 child, and it's an even shot at which gender it becomes. $P(X=1)=0.5$, $P(X=0)=0.5$, and likewise $P(Y=1)=0.5$, so the expected value for boys and girls are both $1*0.5=0.5$.\\\\
            In town B, $P(X=1)=1$, since they refuse to stop having children until a son comes. But since they stop right after, $P(X>1)=0$. This means $E[X] = 1$. For girls, $P(Y=1)=0.5*0.5$ since it means the first birth was a girl, and was immediately followed by a boy. Continuing, $P(Y=2) = 0.5^3$, and so on.\\
            So $E[Y] = 1*0.5^2 + 2*0.5^3 + \ldots = \sum\limits_{i=1}^\infty i*0.5^{i+1} = 1$.\\\\
            Putting it all together, the ratio in town A is $\frac{0.5}{0.5}=\textbf{1}$, and the ratio in town B is $\frac{1}{1} = \textbf{1}$, maintaining the existing ratio in both towns.
        \end{enumerate}
    \end{enumerate}
\end{enumerate}

\end{document}

