\documentclass[12pt,letterpaper]{article}

\newenvironment{proof}{\noindent{\bf Proof:}}{\qed\bigskip}

\newtheorem{theorem}{Theorem}
\newtheorem{corollary}{Corollary}
\newtheorem{lemma}{Lemma} 
\newtheorem{claim}{Claim}
\newtheorem{fact}{Fact}
\newtheorem{definition}{Definition}
\newtheorem{assumption}{Assumption}
\newtheorem{observation}{Observation}
\newtheorem{example}{Example}
\newcommand{\qed}{\rule{7pt}{7pt}}

\newcommand{\assignment}[4]{
\thispagestyle{plain} 
\newpage
\setcounter{page}{1}
\noindent
\begin{center}
\framebox{ \vbox{ \hbox to 6.28in
{\bf CS446: Machine Learning \hfill #1}
\vspace{4mm}
\hbox to 6.28in
{\hspace{2.5in}\large\mbox{Problem Set #2}}
\vspace{4mm}
\hbox to 6.28in
{{\it Handed Out: #3 \hfill Due: #4}}
}}
\end{center}
}

\newcommand{\solution}[4]{
\thispagestyle{plain} 
\newpage
\setcounter{page}{1}
\noindent
\begin{center}
\framebox{ \vbox{ \hbox to 6.28in
{\bf CS446: Machine Learning \hfill #4}
\vspace{4mm}
\hbox to 6.28in
{\hspace{2.5in}\large\mbox{Problem Set #3}}
\vspace{4mm}
\hbox to 6.28in
{#1 \hfill {\it Handed In: #2}}
}}
\end{center}
\markright{#1}
}

\newenvironment{algorithm}
{\begin{center}
\begin{tabular}{|l|}
\hline
\begin{minipage}{1in}
\begin{tabbing}
\quad\=\qquad\=\qquad\=\qquad\=\qquad\=\qquad\=\qquad\=\kill}
{\end{tabbing}
\end{minipage} \\
\hline
\end{tabular}
\end{center}}

\def\Comment#1{\textsf{\textsl{$\langle\!\langle$#1\/$\rangle\!\rangle$}}}



\oddsidemargin 0in
\evensidemargin 0in
\textwidth 6.5in
\topmargin -0.5in
\textheight 9.0in

\begin{document}

\solution{My name}{Hand in date}{Problem set number}{Fall 2014}
% Fill in the above, for example, as follows:
% \solution{Joe Smith}{\today}{1}{Fall 2012}

\pagestyle{myheadings}  % Leave this command alone

\begin{enumerate}
\item SVM
  \begin{enumerate}
  \item [(a)]
    \begin{enumerate}  
    \item [1.]
      $\textbf{w} = [-1,0]^T$\\
      $\theta = 0$
    \item [2.]
      $\textbf{w} = [-0.5,0.25]^T$\\
      $\theta = 0$
    \item [3.]
      I found the two closest positive/negative points, $[(-1.2,1.6), +], [(2,0), -]$, and found the slope between them, $\frac{1.6}{-3.2} = -\frac{1}{2}$, and the midpoint, $(0.4, 0.8)$, so the line with the farthest distance between the two points (the support vectors), has a slope of $2$ with a point $(0.4, 0.8)$, giving the line $y=2x$, which gives $w=[-2,1]^T, \theta=0$.\\\\
      Then, I just minimized $w$ by halving it repeatedly, until I got $w=[-0.5,0.25]$. This $w$ gave $y(w^Tx+\theta)=1$ for both support vectors, so I know this is the smallest value of $w$ I can get.      
    \end{enumerate}
  \item [(b)]
    \begin{enumerate}
    \item [1.]
      $I = \left\{1,6\right\}$
    \item [2.]
      $\alpha = \left\{\frac{5}{32},\frac{5}{32}\right\}$
    \item [3.]
      Objective function value = $\frac{5}{32}$.
    \end{enumerate}
  \item [(c)]
    FINISH ME LATER. $C$ represents how much the SVM should avoid misclassifications. In general, $C$ controls the relative importance of maximizing the margin. For $C=\infty$, we obtain our original hyperplane that we found in (a)-2. For $C=1$, we get a larger margin, with a higher chance of misclassification. The support vectors for $C=1$ can now be inside the margins. For $C=0$ has an even wider margin, with even larger misclassification. (FINISH ME LATER)
  \end{enumerate}
\item Kernels
  \begin{enumerate}
    \item[(a)]
      \begin{algorithm}
        1. Initialize $\alpha$ to $\vec{0}$ of length $n$, where $n$ is the number of examples.
        2. Initialize $\theta$ to $0$.
        3. 
      \end{algorithm}
  \end{enumerate}
\item Answer to problem 3
\end{enumerate}

\end{document}

